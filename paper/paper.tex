\documentclass[runningheads]{llncs}

\usepackage{graphicx}
\usepackage{proof}
\usepackage{stmaryrd}
\usepackage{amsmath}
\usepackage{amssymb}

% From https://tex.stackexchange.com/questions/235118/making-a-thicker-cdot-for-dot-product-that-is-thinner-than-bullet
\makeatletter
\newcommand*\bigcdot{\mathpalette\bigcdot@{.5}}
\newcommand*\bigcdot@[2]{\mathbin{\vcenter{\hbox{\scalebox{#2}{$\m@th#1\bullet$}}}}}
\makeatother

\newcommand{\labinfer} [3] [] {\infer[{\textsc{#1}}]{#2}{#3}}
\newcommand{\eq} {\stackrel{\bigcdot}{=}}

\newcommand{\rewrite} {\ensuremath{\leadsto}}
\newcommand{\ra} {\rightarrow}
\newcommand{\sem} [1] {\llbracket #1 \rrbracket}
\newcommand{\Lsem} [2] {\mathcal{L}_{#1}\sem{#2}}

\newcommand{\Expr} {\textnormal{Expr}}

\newcommand{\keyword} [1] {\textbf{#1}}

\newcommand{\here} {\keyword{here}}
\newcommand{\at} {\keyword{at}}

\newcommand{\todo} [1] {\textbf{TODO}: #1}

\newcommand{\tentail} {\rhd}


\begin{document}

\title{Egret: A Tool For Equational Reasoning}
\maketitle

\section{Abstract}

\section{Introduction}

Lorem ipsum~\cite{Elliott-2018-ad-extended}.

\section{Type System}

\subsection{Object language}
\[
  \begin{array}{c}
    \labinfer[Ty-Var]{\Gamma \vdash x : \alpha}{(x : \alpha) \in \Gamma}
    \\\\
    \labinfer[Ty-App]{\Gamma \vdash f(e) : \beta}{\Gamma \vdash f : \alpha \ra \beta & \Gamma \vdash e : \alpha}
  \end{array}
\]

\subsection{Equations}

\[
  \begin{array}{c}
    \labinfer[Ty-Eq]{\Gamma \vdash e_1 \eq e_2 : \alpha}{\Gamma \vdash e_1 : \alpha & \Gamma \vdash e_2 : \alpha}
  \end{array}
\]

\section{Semantics}
\todo{This seems a lot like the theory of explicit substitutions. Can we use this?}

A \textit{theory} is a pair $(\Gamma, E)$ where $\Gamma$ is a consistent type assignment and $E$ is a set
of equations that are well-typed under $\Gamma$. That is, if $\mathcal{T} = (\Gamma, E)$ is a theory then for every $(e_1 \eq e_2) \in E$ we have $\Gamma \vdash e_1 \eq e_2 : \alpha$ for
some type $\alpha$.

\subsection{Basic Rules}
\[
  \begin{array}{c}
    \labinfer[Theory-Typing]{(\Gamma, E) \tentail e : \alpha}{\Gamma \vdash e : \alpha}
    \\\\
    \labinfer[Eq-Project]{(\Gamma, E) \tentail e_1 \eq e_2}{(e_1 \eq e_2) \in E & \Gamma \vdash e_1 \eq e_2 : \alpha}
    \\\\
    \labinfer[Refl]{\mathcal{T} \tentail e \eq e}{\mathcal{T} \vdash e : \alpha}
    ~~~
    \labinfer[Sym]{\mathcal{T} \tentail e_2 \eq e_1}{\mathcal{T} \tentail e_1 \eq e_2}
    ~~~
    \labinfer[Trans]{\mathcal{T} \tentail e_1 \eq e_3}{\mathcal{T} \tentail e_1 \eq e_2 & \mathcal{T} \tentail e_2 \eq e_3}
    \\\\
    \labinfer[Rewrite]{\mathcal{T} \tentail e_1 \rewrite e_2}{\mathcal{T} \tentail e_1 \eq e_2}
  \end{array}
\]

\subsection{Denotational Semantics}

Consider a denotation function for the object language $\mathcal{M}(\cdot) : \Expr \ra D$. We
say that this is a \textit{model} for $\mathcal{T}$ if the following property holds:

\[
  \begin{array}{c}
    (\mathcal{T} \tentail e_1 \eq e_2) \implies \mathcal{M}(e_1) = \mathcal{M}(e_2)
  \end{array}
\]

\noindent
and we write
\[
  \mathcal{M} \models \mathcal{T}
\]

% We require a denotation function for the object language $\Lsem{T}{\cdot} : \Expr \ra D$ that is invariant
% under the defined rewrites:
%
% \[
%   (T \vdash e_1 \rewrite e_2) \implies (\Lsem{T}{e_1} = \Lsem{T}{e_2})
% \]
%
% % Given a denotation function for the object language $\Lsem{\cdot} : \Expr \ra D$, we define:
% %
% % \[
% %   \begin{array}{c}
% %     \sem{R} : \Expr \ra D
% %     \\
% %     \sem{\here(x \rewrite y)}(x) = \Lsem{y}
% %     \\
% %     \sem{\at(0, x \rewrite y)}(x) = \Lsem{y}
% %   \end{array}
% % \]
%
\noindent
We can then give the denotation function for rewrites $\sem{\cdot}$

\[
  \begin{array}{c}
    \sem{e_1 \rewrite e_2} : \Expr \ra \mathcal{P}(\Expr)
    \\
    \sem{e_1 \rewrite e_2}(e_1) = \{e_1, e_2\}
    \\
    \sem{e_1 \rewrite e_2}(e_3) = \{e_3\}\textnormal{ where $e_1\neq e_3$}
    \\
    \sem{e_1 \rewrite e_2}(f(e)) = \{ f'(e') \mid f' \in \sem{e_1 \rewrite e_2}(f) \land e' \in \sem{e_1 \rewrite e_2}(e) \}
        \textnormal{ where $e_1 \neq f(e)$}
  \end{array}
\]

\noindent
This function is extended to operate on entire theories:

\[
  \begin{array}{c}
    \sem{\mathcal{T}}(e) = \bigcup\{ \sem{e_1 \rewrite e_2}(e) \mid \mathcal{T} \rhd e_1 \rewrite e_2 \}
  \end{array}
\]

\noindent
It follows that $\sem{\cdot}$ respects models for $\mathcal{T}$. This is \textit{soundness}.
\[
  \forall e' \in \sem{\mathcal{T}}(e).\; \mathcal{M}(e) = \mathcal{M}(e')
\]

\noindent
The converse of this, \textit{completeness}, does not always hold. When it does hold, it shows that the theory $\mathcal{T}$ captures \textit{every} equality that holds in the model $\mathcal{M}$.
\[
  \forall e', e.\; \mathcal{M}(e) = \mathcal{M}(e') \implies e' \in \sem{T}(e)
\]
%
% \noindent
% This satisfies the following property
%
% \[
%   T \vdash e_1 \rewrite e_2 \implies \forall b \in \sem{e_1 \rewrite e_2}(a),\; \Lsem{T}{a} = \Lsem{T}{b}
% \]


\subsection{Operational Semantics}

\subsection{Soundness}

\subsection{Completeness}


\bibliographystyle{splncs04}
\bibliography{paper}

\end{document}

